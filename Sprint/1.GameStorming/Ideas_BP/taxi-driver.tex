\documentclass[11pt,twoside,a4paper]{article}
\usepackage[english]{babel} %English hyphenation
\usepackage{amsmath} %Mathematical stuff
\usepackage{amsthm}
\usepackage{amssymb}

%Hyperreferences in the document. (e.g. \ref is clickable)
\usepackage{hyperref}

%Pseudocode
\usepackage{algorithm}
\usepackage[noend]{algpseudocode}
%You can also use the pseudocode package. http://cacr.uwaterloo.ca/~dstinson/papers/pseudocode.pdf
%\usepackage{pseudocode}

\usepackage{a4wide,times}
\title{Game Concept: Taxi Trouble} %The title, e.g. Algoritmiek Assignment 1
\author{
        Joost van Oorschot, jjevanoorschot, 4220471
}
\begin{document}

\maketitle
%\clearpage

%It is possible to automatically generate a Table of Contents or a Table of Algorithms and more.
%\tableofcontents
%\listofalgorithms

\section*{Context}
Two or four people are sitting together, which could be at a waiting room, in a train, or at a party. All players have a mobile phone on which they play the game. Having all players in the same room allows for verbal communication, so that interaction is natural and a feeling of community is formed.

\section*{The premise}
A taxi driver has to get around town as quickly as possible, avoiding other traffic and accumulating as much revenue as possible. At the same time, the taxi driver is competing with another taxi driver. The taxi driver that ends with the most money wins the game.

\section*{Gameplay}
The game can be played with either one or two teams. Each team is composed of 2 players. One of the players is the taxi driver, and the other is the navigator. The taxi driver is responsible for navigating the car: dodging traffic, making sharp turns and stopping next to customers. It is the navigators job to look at the map and to determine the optimal route for picking up passengers and to tell the taxi driver when to make a turn. Because of the fast paced nature of the game both roles are very hectic, and require good communication between both players.

\end{document}
