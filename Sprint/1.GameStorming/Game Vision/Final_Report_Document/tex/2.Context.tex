In this section a description and research will be presented for which the game is developed. In many situations a group of people have to wait together at the same place for a fixed period of time. This group may have gathered for a large event and most of them might not even know each other. As unpleasant waiting experiences might lead to more negative service evaluations as described by Yan and Lotz (2006) it is important to have a way of entertaining such groups to improve their waiting experience. A challenging way of doing this is by creating an interactive, competitive and collaborative computer game that can be played by all present persons together. This corresponds to Sellar (2004) indicating that playing computer games is not simply a form of entertainment for individuals anymore. Instead collaboration takes an increasingly larger role in playing computer games. Khaled et al. (2009) point out that a possible way to raise collaboration is by dividing tasks within a game that can be done in parallel by multiple players with one shared goal. The game Taxi Trouble does this by dividing the tasks for each team by assigning one player to drive the taxi while the other navigates the player to the customers. This is as well confirmed by Ducheneaut and Moore (2004) as they state that what makes a game entertaining and stand out for many players are the collaborative aspects of the activities, the shared experience of playing the game and the satisfaction of socializing with a group of people.