In this section a description and research will be presented for which the game is developed. In many situations a group of people have to wait together at the same place for a fixed period of time. This group may have gathered for a large event and most of them might not even know each other. As unpleasant waiting experiences might lead to more negative service evaluations as described by Yan and Lotz (2006) it is important to have a way of entertaining such groups to improve their waiting experience. A challenging way of doing this is by creating an interactive, competitive and collaborative computer game that can be played by all present persons together. This corresponds to Sellar (2004) indicating that playing computer games is not simply a form of entertainment for individuals anymore. Instead collaboration takes an increasingly larger role in playing computer games. Khaled et al. (2009) point out that a possible way to raise collaboration is by dividing tasks within a game that can be done in parallel by multiple players with one shared goal. The game Taxi Trouble does this by dividing the tasks for each team by assigning one player to drive the taxi while the other navigates the player to the customers. This is as well confirmed by Ducheneaut and Moore (2004) as they state that what makes a game entertaining and stand out for many players are the collaborative aspects of the activities, the shared experience of playing the game and the satisfaction of socializing with a group of people. Further the research of Gajadhar et al. (2008) made clear that playing games together while being at the same place results in more fun, perceived competence, and challenge being experienced then playing far away from each other. This benefits the context as the players of the game will be waiting together at the same place.\qquad Focussing on the competitive aspects of games Vorderer (2003) states that including competitive elements allows for more active player engagement and for direct feedback on the players' actions. From research of Cairns et al. (2013) regarding competition in games it follows that playing a game against another person results in more immersion than playing against a computer, but though there is no remarkable difference in immersion when the other player(s) are at the same place or not. Finally as explained by Granic et al. (2013) games offer a way of having control but at the same time just enough unpredictability so that the players can feel satisfaction and pride when a goal is reached after effort. In Taxi Trouble this satisfaction is achieved after a team has successfully managed to deliver a customer at its destination.