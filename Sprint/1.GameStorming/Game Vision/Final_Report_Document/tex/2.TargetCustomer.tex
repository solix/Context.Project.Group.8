First of all an analysis will be given of the target market of the game. The game Taxi Trouble is focussed on entertaining large groups of people waiting together on large events like festivals. Players of the game are supposed to form pairs each competing against all other pairs that are joining the game. In the game verbal communication and the ability to see each other is essential as teamplay is the key to winning the game. Further the game is targeted on players equipped with a smart device such as a smartphone or tablet-pc with \textit{Android} as operating system , or players that are at least familiar with the use of a smart device.\\
The target audience of the game consists of relatively young people with a well-developed responsiveness and insight. This means that in a game the pair of players for which these skills are developed the best will have the best chances of winning the game. The age category for people that are most suited to like the game the most is therefore estimated to consist of people between the age of 12 and 40 years. Further the target customers consist of both men and women, so the game is enjoyable for both genders. Also the cultural and educational background of the players is not of major influence on the capability to play the game, though it might be helpfull to have a basic knowledge of the English language.
     