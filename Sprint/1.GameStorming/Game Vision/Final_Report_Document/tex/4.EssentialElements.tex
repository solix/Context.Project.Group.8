

This next section describes the elements of the game that are essential for fulfilling the user needs by giving a description of the most vital elements of the game. Taxi Trouble takes place in a city. Because of the low amount of people using a taxi, this city has a lot of rivaling taxi companies. The situation has gone so bad, that taxi's actually chase each other to steal passengers. The game is played in teams. Every team has one taxi. The goal of each team is to serve as many passengers as possible.


\subsection{Competitive Elements}
There will always be less passengers available than there are competing taxis in the game. This means that, if you want to drive a passenger, you have to get your taxi to him before all other taxis. Once one taxi has reached a passenger, it has to bring the passenger to its destination. However, other cabs can try to steal the passenger by bumping into the taxi in which the passenger is sitting. The cab that eventually gets the passenger to the destination on time, earns points. This gameplay element results in all empty taxis chasing the taxis transporting passengers. This chase is essential because of it competitive nature. It fulfills the customers' need for competition. 

\subsection{Random Elements}

Allong the way of chasing and being chased, taxis can collect power-ups on the road. These power-ups will enable the taxi to perform a special action once activated. A simple example of a power-up would be a speed boost, temporarily increasing the speed of the taxi. Another example could be temporary invisibility, which could be used to escape a car chase or to sneak up on a taxi that has a passenger on board. These power-ups are randomly distributed throughout city. This fulfills the customers' need for an unknown factor in the game. 

\subsection{Co-operative Elements}
The game fulfills the customers' need for co-operation through the emphasis on communication in the teams. Each team consists of two players: one Driver and one Navigator. The Driver controls the taxi. It is his job to maneuver through the traffic to bring passengers to their destination. The Driver can only see the direct vicinity of the taxi. This means he can travel down roads, but he has no way of knowing where he is going. This is where the Navigator comes in. The Navigator has a live map of the city that displays the current location of the taxi, the location of power-ups and the location (or destination) of the passengers. It is the Navigator's job to guide the Driver to the right locations. This leads to a need to communicate. The Navigator is also the only player in the team that can activate power-ups. For a power-up like a speed boost, for which the time of activation can be crucial, it is essential that the Navigator and the Driver are co-operating. As a consequence of the aforementioned game mechanics, each pair is forced to communicate as well as possible to win from the pairs they are competing with.


