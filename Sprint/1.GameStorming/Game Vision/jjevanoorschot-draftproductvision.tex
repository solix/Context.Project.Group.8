\documentclass[11pt,twoside,a4paper]{article}
\usepackage[english]{babel} %English hyphenation
\usepackage{amsmath} %Mathematical stuff
\usepackage{amsthm}
\usepackage{amssymb}

%Hyperreferences in the document. (e.g. \ref is clickable)
\usepackage{hyperref}

%Pseudocode
\usepackage{algorithm}
\usepackage[noend]{algpseudocode}
%You can also use the pseudocode package. http://cacr.uwaterloo.ca/~dstinson/papers/pseudocode.pdf
%\usepackage{pseudocode}

\usepackage{a4wide,times}
\title{Game Concept: Taxi Trouble} %The title, e.g. Algoritmiek Assignment 1
\author{
        Joost van Oorschot, jjevanoorschot, 4220471
}

\begin{document}
\maketitle
\section*{Gameplay}
\begin{enumerate}
\item One round of the game will last two minutes, with a countdown timer visible on the screen. Players can choose between one round, three rounds and five rounds per game. The team with the most rounds won win the game.

\item The player can set the amount of passengers that have to be delivered to win a round. The player can also set the amount of rounds that take place.

\item The map will be sufficiently large to hold 4 Taxi's, but small enough so that Taxi's will frequently meet. 

\item Taxi's will be able to drive freely through the map, crashing into each other to steal each other's customers.

\item To pick up a customer, a Taxi simply drives over the customer, and he will hop in the car. 

\item Pedestrians will walk around the map, but the player will not be able to driver on the curb.

\item Traffic will spawn randomly throughout the city, and the Taxi driver has to avoid the incoming traffic.

\item The navigator has an overview of the map, with possible passengers and the other Taxi cars lighting up.

\item Players will be able to select the appearance of their car. They can also place decorative items on the car.
\end{enumerate}
\section*{Product vision}
\begin{enumerate}
\item Our target audience is smartphone owners between $10$ and $25$ years old.
\item Customer needs:
\begin{enumerate}
\item To be entertained for a short period of time.
\item To easily pick up the game.
\item To be able to play the game with a large group.
\end{enumerate}
\item Crucial to the needs is a selection of fun gameplay mechanics, and at least 4 player multiplayer, but preferably with more players.
\item The game "Taxi Driver 2", available in the Play Store is somewhat similar.
\item The timeframe is $10$ weeks. If we wish to publish the game to Google Play, then we will need to pay $25$ euro's. That is our total budget.
\end{enumerate}
\end{document}
