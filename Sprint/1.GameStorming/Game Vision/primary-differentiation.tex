\documentclass[11pt,twoside,a4paper]{article}
\usepackage[english]{babel} %English hyphenation
\usepackage{amsmath} %Mathematical stuff
\usepackage{amsthm}
\usepackage{amssymb}

%Hyperreferences in the document. (e.g. \ref is clickable)
\usepackage{hyperref}

\usepackage{a4wide,times}
\title{Game design draft} 
\author{
        Joost van Oorschot, jjevanoorschot, 4220471
}

\begin{document}
\maketitle
\section*{Primary Differentiation}
\subsection*{Introduction}
The product that we are developing has a number of properties that differentiates it from alternatives already on the market. This differentiation takes place on various levels, from a very conceptual level to a technical level. The following passage will serve to highlight the most differentiating aspects of our product, to explain what exactly makes these aspects different from existing products and why this is significant.

\subsection*{Comparison to non-computer games}
Before making the obvious comparison between the product and other computer games, it is worth considering that the product can also be compared to non-computer games. A very important aspect of the game design is that the game will be played by a group of people that have to be in the same room and have to interact verbally. This aspect of the game design makes the game setting and the game dynamics very similar to traditional games, such as board games. The main point of differentiation from traditional games is that while the human interaction is similar, the gameplay possibilities are virtually endless due to the virtually endless capabilities of computers. The gampeplay of our game would simply be impossible to implement in a non-computer game.

\subsection*{Comparison to computer games}
There are two computer games that share some features with our game. The first is \emph{Taxi Driver.  2}\footnote{\url{https://play.google.com/store/apps/details?id=manastone.game.Taxi.GG}}. \emph{Taxi Driver 2} is mainly similar because our game shares the same premise, namely: the player controls a taxi and has to pick up and drop off passengers. There are quite a few points of differentiation: our game will support multiplayer, feature much more fast-paced gameplay, and will have a driver and a navigator working together.

The second game that shows similarities to our game is \emph{Spaceteam}\footnote{\url{https://play.google.com/store/apps/details?id=com.sleepingbeastgames.spaceteam}}. \emph{Spaceteam} is similar because the game shares an important gameplay mechanic with our game. Both games have two players in a team that cannot win without communicating verbally with each other. In \emph{Spaceteam} this comes in the form of shouting Startrek-like commands to each other, and in our game this consists of the navigator shouting directions to the driver. Where the games differ is in basically all other aspects. Apart from the mentioned similarity there is no overlap in gameplay, nor is there in theme.

\subsection*{Technical aspects that differentiate}
\emph{Taxi Trouble} differs from most other Android games in the way that it handles multiplayer. There are very few real-time multiplayer games for mobile, and even less that support $4$ or more players. Implementing the multiplayer of our game will be one of the most difficult technical challenges of the project. At the same time, the fast-paced real-time multiplayer is perhaps the best differentiation point of the game, offering a gameplay experience that is usually only offered on non-mobile gaming platforms.
\end{document}
