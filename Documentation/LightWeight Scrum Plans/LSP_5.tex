\documentclass{article}
\usepackage{xcolor}
\usepackage{fullpage}
\usepackage{graphicx}
\usepackage{epstopdf}
\usepackage{caption}
\usepackage{verbatim}
\usepackage{amssymb}
\usepackage{amsmath,amsthm}
\usepackage{amsfonts}
\usepackage{enumerate}
\usepackage{enumitem}
\usepackage{listings}
\usepackage{qtree}
\usepackage{tikz}
\usepackage{bm}
\usepackage{frame,color}
\usepackage{datetime}
\usepackage{etoolbox}
\usepackage{emerald}
\usepackage[T1]{fontenc}
\makeatletter
\patchcmd{\chapter}{\if@openright\cleardoublepage\else\clearpage\fi}{}{}{}
\makeatother
\definecolor{shadecolor}{rgb}{184,184,184}
\lstset{language=Matlab,
        frame = single,
        breaklines=true
}
\usetikzlibrary{calc,positioning}
\usetikzlibrary{arrows,automata}
\renewcommand{\familydefault}{\sfdefault}

\begin{document}
%%%%%%%%%%%%%%%%%%%%%%%%%%%%%%%%TITLE PAGE%%%%%%%%%%%%%%%%%%%%%%%%%%%%%%%%%%%%%%%%%%%%%%%%
\begin{figure}
    \begin{minipage}[H]{0.33\textwidth}
		\vspace{0.3cm}
		\includegraphics[scale=0.8]{images/TUDelftLogo.eps}
	\end{minipage}
	\begin{minipage}[H]{0.33\textwidth}
		\begin{center}
			\ECFJD{Context Project\\Computer Games Group 8}
		\end{center}
		\begin{center}
			\includegraphics[scale=0.8]{images/Lg.eps}	
		
		\end{center}
	\end{minipage}
	\begin{minipage}[H]{0.425\textwidth}
			\begin{flushright}
				\small{Rob van Bekkum \qquad rvanbekkum\qquad 4210816}\\
				\small{Thijs Brands \qquad tlmbrands\qquad 4247132}\\
				\small{Soheil Jahanshahi \qquad sjahanshahi\qquad 4127617}\\
				\small{Aidan Mauricio \qquad amauricio\qquad 4195175}\\
				\small{Joost van Oorschot\qquad jjevanoorschot\qquad  4220471}\\
				\vspace{0.2cm}
				\small{\textbf{Date:} 23 May 2014}
			\end{flushright}
	\end{minipage}
\end{figure}

\begin{minipage}[H]{\textwidth}
\vspace{0.3cm}
		\begin{center}
		\vspace{0.3cm}
			\Huge{\textbf{Lightweight Scrum Planning 5}}\\
		\vspace{0.3cm}	
		\vspace{0.7cm}		
		\end{center}
	\end{minipage}

\section*{Set of Backlog items}
The selected items of the backlog that we will be working on this comming week  are (ordered by importance):
\begin{itemize}
	\item Multiplayer architecture
	\item User testing
	\item Better graphics
	\item Initial power-ups	
\end{itemize}

\section*{User stories}
\subsection*{Multiplayer architecture}
As a user\\
Given that I created a lobby\\
I will become host\\\\
As a user\\
Given that my game did not receive the location of a car from the host\\
I will see that car at the predicted location
\subsection*{User testing}
As a tester\\
I want a to play a playable version of the game\\\\
As a tester\\
Given that I have played the game\\
I will take part in a Q\&A session
\subsection*{Better graphics}
As a user\\
I want a visually pleasing game map
\subsection*{Initial power-ups}
As a navigator\\
Given that there is a power-up on the map\\
Then I can use that power-up to influence the game
\section*{List of tasks, assigned group members and estimation of effort}
\textbf{Multiplayer architecture} \\
\begin{tabular}{ | l | l | l | }
\hline
\textbf{Task} & \textbf{Assigned to} & \textbf{estimated effort} \\ \hline
Client / server architecture & Thijs \& Joost & 4 points \\ \hline
Client side prediction & & 4 points \\\hline
Entity interpolation &  & 5 points \\ \hline
Server client synchronization & & 5 points \\ \hline
\end{tabular} \newline
\newline \\
\textbf{User testing} \\
\begin{tabular}{ | l | l | l | }
\hline
\textbf{Task} & \textbf{Assigned to} & \textbf{estimated effort} \\ \hline
Formulate interview questions &  & 2 points \\ \hline
Conduct user testing & & 2 points \\\hline
\end{tabular} \newline
\newline \\
\textbf{Better graphics} \\
\begin{tabular}{ | l | l | l | }
\hline
\textbf{Task} & \textbf{Assigned to} & \textbf{estimated effort} \\ \hline
New map textures &  & 2 points \\ \hline
\end{tabular} \newline
\newline \\
\textbf{Initial power-ups} \\
\begin{tabular}{ | l | l | l | }
\hline
\textbf{Task} & \textbf{Assigned to} & \textbf{estimated effort} \\ \hline
Power-up textures &  & 2 points \\ \hline
Activation of power-ups in navigator view & & 3 points \\ \hline
Two simple power-ups  & & 2 points \\ \hline
\end{tabular} \newline
\newline \\
\end{document}