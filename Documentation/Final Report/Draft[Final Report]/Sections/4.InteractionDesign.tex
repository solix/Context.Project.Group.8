After seven weeks of development we conducted a user test among visitors of the TU Delft Science Center. This section will describe the Interaction Design aspects of the usability evaluation that we conducted there. Firstly, we will discuss our evaluation methods and what part of the system we tested. Secondly, we will give an overview of how the testing was done, by discussing the setting and location of the user test, a description of the users that tested the game, and the methods that we used during the test. Lastly, we will give a summary of our findings.
\\\\
For the usability evaluation, we have chosen to use the empirical 'experiment' practice. An experiment was best suited for our user test, because we wanted to observe users interact with the game in order to discover flaws in the game, and to identify gameplay elements that were not considered fun. A big aspect of Taxi Trouble is communication, which resulted in users practicing Think Aloud without being asked by us. This was very useful for our evaluation of the user test, as it gave us a lot of information about how the users perceived the game.

The usability evaluation was done right before the release of the beta release of Taxi Trouble, so that we could incorporate our findings into the beta version. This means that we evaluated the usability of the alpha version of Taxi Trouble. The alpha version was missing a lot of features compared to the final version, but it was stable enough to conduct a user test with.
\\\\
For the setting of the user test we chose the TU Delft Science Center. The Science Center is a good location for testing because it receives a lot of visitors that fit into the user demographic of Taxi Trouble, and there are sufficient facilities for conducting a user test. We were appointed a large, open room, with two racing chairs in the center of the room. The racing chairs made the user test more fun for the younger users, and fit the theme of Taxi Trouble well.

The users that tested Taxi Trouble ranged in age from $8$ to $24$. We took great care during the user tests with the younger users. The parents of the younger users were present during the user tests at all times and we made sure to mention that the users could stop the test at any time.
\\\\
The user test was performed in groups of two users. At the start of the test we asked the users to sit next to each-other and to pick a role in the game (either navigator or driver). Then, we let the users play the game for about five minutes. During this time we logged what the users said to each-other and we did not interact with the users. After playing the game, we interviewed the users about the game, asking open questions about the art style of the game, the controllability of the game, and about the gameplay.
\\\\
We did a formative evaluation of the test resuslts and we will give a summary of our findings in this section. Overall, the users were very pleased with our game. Even though all users stated that they found the controllability of the game sufficient, we noticed that most users struggled to understand what the buttons to control the car meant, and they had to take some time to figure that out by trial-and-error. This resulted in us creating buttons that mimicked the look of an actual gas pedal and brake.

During the evaluation it became apparent that users did not immediately understand the navigator view. To make the navigator screen more understandable, we created a HUD that shows to which team the navigator belongs.

Another thing that we found out, is that when a taxi is carrying a passenger, it is hard for the users to distinguish the front of the taxi from the back. In response to this we created new sprites for the taxi that clearly show the difference between the front and back of the car.

The results of the usability evaluation helped us identify a number of flaws, which allowed us to fix them. Additionally, we were able to pinpoint which gameplay features the users liked and which features they disliked.