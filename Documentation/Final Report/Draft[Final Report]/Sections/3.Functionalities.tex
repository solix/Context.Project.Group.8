In this section the important functionalities and key features that are developed for Taxi Trouble  will be explained. Each functionally is an important part of the game and enhances the experience of the user. Functionalities are classified under four categories: the visual aspect, the game model, the sound and the multiplayer. 

\subsection*{Visual Aspect} % (fold)
\label{sub:visual_aspect}
The visual aspect of the game relates to how the game appears to the players. It focuses on the appearance of the main game world and the components that are used to enhance the game for the player. \\

\textbf{The driver's screen }has an myopic 2D view of the map. It makes the driver's sight on the map limited so that it makes it impossible to win the game without the navigator's cooperation. There are touchscreen controls that can be used by the driver to control the taxi's acceleration and steering angle.\\   

\textbf{The navigator's screen }has a 2D top-down view of the map. The navigator can zoom and pan to explore the game world.  The navigator can use his overview of the game to guide the taxi driver. There is a button provided on the navigator's screen, which can be used to activate power-ups that are picked up by the driver.\\

\textbf{The head-up display }is part of the graphical user interface. It is used to show information to the player during gameplay.  The following information is represented textually:
			\begin{itemize}
				\item The team number, which indicates to which team the player belongs. 
				\item A scoreboard that represents the score achieved by the team. 
				\item The time left until the game ends.  
				\item The time remaining for the driver to drop  off a passenger to his/her destination. 
			\end{itemize}	

\textbf{The main menu }is the first interactive screen shown after starting the game. It provides two buttons: a play button and a leaderboard button. The player can start the game by pressing play. The player can view the top scores of the current day or week, or the all-time top scores by pressing the leaderboard button.\\

%%%%%%%%%%%%%%%%%%%%%%%%%%%%%%%
%%%% THIS SECTION NEEDS TO BE REWRITTEN%%%%%%%%%
%%%%%%%%%%%%%%%%%%%%%%%%%%%%%%%

\subsection*{Game model}
	The game model is the main aspect of the game. It contains instances designed and developed to influence the game state while events are triggered by the user. Here are some of the most important game models explained.

\begin{itemize}
	
	\item \textbf{Teams }are made of two participants where each will takes a distinct role in the game; Either as a driver or navigator.
	\item \textbf{Taxi} is a solid object, It consists of wheels that is controlling the steering and acceleration. When two taxis collide with each other, collision is detected and natural reflection of the taxi body upon collision is performed.  	
	\item \textbf{Passenger} are spawned in random locations on the map. They are meant to be picked up by the driver and to be dropped off by the teams to their destinations. Whenever a passenger is dropped off to his/her destination, a new passenger is then re-spawned in a random location on the map. Passengers can also be stolen by other taxi while they are on their way to destination. When a passenger is picked up by the taxi driver, the driver needs to drop off passenger before the drop off timer is ended.
			
	\item \textbf{Power-ups } of are three types. Invincibility, speed boost, increase drop off timer.`Invincibility' is made when taxi wants to protect the passenger from getting stolen. `Speed boost' is there to increase the acceleration of the taxi vehicle for 10 seconds . `Increase drop off' will extra 10 second to the time that needs to be dropped off by passenger.\\
	These power ups can be picked up on the map by taxi driver and can be activated by the navigator during the game play.   
\end{itemize}	

%%%%%%%%%%%%%%%%%%%%%%%%%%%%%%%%

\subsection*{Sound}
	Having good sound effects can create a positive impact on user's experience. Sound effects are designed to immerse the player into the virtual game world, making the game more entertaining and satisfying.\\
	In the game, sound effects are used when certain events take place. These events are car collision, passenger pick up and drop off, power up pick up and activation, and passenger stealing.


\subsection*{Multiplayer}
One of the primary requirements for the design of a multiplayer game is to develop a sense of shared space among the players. The architecture that is chosen is the peer to peer(p2p) architecture. \\
A multiplayer game can be started by inviting friends or choosing to be joined with random players. Once enough players have joined a game room, the game will commence. The game can be played with four, six our eight players. During the game, the illusion of a shared space is maintained through network messages which are being broadcasted between players. These messages contain information about the game model and tell the receiving client to update its own model accordingly.
