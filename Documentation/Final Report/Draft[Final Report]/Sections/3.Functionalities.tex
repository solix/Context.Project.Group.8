In this section the important functionalities and key features that are developed for Taxi Trouble  will be explained. Each functionally is an important part of the game and enhances the experience of the user. Functionalities are classified under four categories: the visual aspect, the game model, the sound and the multiplayer. 

\subsection*{Visual Aspect} % (fold)
\label{sub:visual_aspect}
The visual aspect of the game relates to how the game appears to the players. It focuses on the appearance of the main game world and the components that are used to enhance the game for the player. \\

\textbf{The driver's screen }has an myopic 2D view of the map. It makes the driver's sight on the map limited so that it makes it impossible to win the game without the navigator's cooperation. There are touchscreen controls that can be used by the driver to control the taxi's acceleration and steering angle.\\   

\textbf{The navigator's screen }has a 2D top-down view of the map. The navigator can zoom and pan to explore the game world.  The navigator can use his overview of the game to guide the taxi driver. There is a button provided on the navigator's screen, which can be used to activate power-ups that are picked up by the driver.\\

\textbf{The head-up display }is part of the graphical user interface. It is used to show information to the player during gameplay.  The following information is represented textually:
			\begin{itemize}
				\item The team number, which indicates to which team the player belongs. 
				\item A scoreboard that represents the score achieved by the team. 
				\item The time left until the game ends.  
				\item The time remaining for the driver to drop  off a passenger to his/her destination. 
			\end{itemize}	

\textbf{The main menu }is the first interactive screen shown after starting the game. It provides two buttons: a play button and a leaderboard button. The player can start the game by pressing play. The player can view the top scores of the current day or week, or the all-time top scores by pressing the leaderboard button.\\

\subsection*{Game model}
	The game model lies at the heart of the design of the game. The model consists of all entities that populate the game. In this section we will give an overview of the most important entities that make up Taxi Trouble's model.

\begin{description}
	\item[Taxi] \hfill \\
	 The taxi entity is the most important entity in the game, as one would guess by Taxi Trouble's name. A taxi contains a body and wheels, which are separate entities on their own. Each taxi also has a team assigned to it. Furthermore, the taxi entity contains a set of properties such as the passenger that is in the taxi and its current speed, maximum speed, angle, and acceleration.
	\item[Team] \hfill \\
	A team is made of a driver and a navigator. Each team has a corresponding taxi and the team contains the state that the team is in. The state of the team is made up of the team score, whether the team owns a power-up and the appearance of the team.
	\item[Passenger] \hfill \\
	Passengers are spawned at random locations on the map. They are meant to be picked up by the driver and to be dropped off at their destinations before a drop off timer runs out. Whenever a passenger is dropped off at its destination, a new passenger is spawned at a random location on the map. Passengers can be stolen by other taxis while they are on their way to their destination when taxis bump into each other.		
	\item[Power-ups] \hfill \\
	Power-ups are randomly spawned on the map and they give a driver a temporal gameplay advantage. The driver can pick up a power-up, after which the navigator has to activate the power-up. There are three types of power-ups, which are invincibility, a speed boost, and an increase for the drop off timer. The invincibility power-up prevents a taxi against passenger stealing. The speed boost power-up increases a taxi's speed for ten seconds. Lastly, the timer increasing power-up grants an extra ten seconds of drop-off time to a taxi.
\end{description}	

\subsection*{Sound}
	Having good sound effects can create a positive impact on user's experience. Sound effects are designed to immerse the player into the virtual game world, making the game more entertaining and satisfying.\\
	In the game, sound effects are used when certain events take place. These events are car collision, passenger pick up and drop off, power up pick up and activation, and passenger stealing.


\subsection*{Multiplayer}
One of the primary requirements for the design of a multiplayer game is to develop a sense of shared space among the players. The architecture that is chosen is the peer to peer(p2p) architecture. \\
A multiplayer game can be started by inviting friends or choosing to be joined with random players. Once enough players have joined a game room, the game will commence. The game can be played with four, six our eight players. During the game, the illusion of a shared space is maintained through network messages which are being broadcasted between players. These messages contain information about the game model and tell the receiving client to update its own model accordingly.