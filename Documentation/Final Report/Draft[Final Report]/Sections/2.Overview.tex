For the past nine weeks our game development team has been developing a competitive, cooperative and interactive computer game designated with the name Taxi Trouble. In this chapter an overview is given of the developed product and its implemented functionalities. A more elaborate description of the functionalities of the game is given in the succeeding chapter.
\\*
Taxi Trouble is a game that can be played together by 4 to 8 players at the same time. The game is played on an Android device and uses the Google Play Game Services to allow people to connect and play the game with each other. A player can select the amount of players that he or she wants to play a game with after which a waiting room is initiated. When all players are connected, the game starts and teams are formed randomly. Each team consists of two players where one gets the role of taxi-driver and the other gets the role of navigator. The taxi-driver of each team drives around in a large city and has only a limited view of his surroundings. The navigator of a team has a complete overview of the citymap and needs to verbally guide its partner to the right locations. The performance of a team in the game will depend on the cooperation whitin said team.
\\*
When the game starts the main menu of the game is shown in which the player can choose to either press the start button, to start a new game, or the leaderboard button to check out the leaderboard.
To start a game the player logs in with his/her Google Account and chooses the amount of players with which the player wants to play. A player can also invite other players with their Google+ Account. When everyone has joined the game the game starts and the roles are divided.
As explained earlier each team consists of a taxi-driver and navigator which have to communicate very well to win the game. Each driver has a taxi which he or she can steer and accelerate. There are passengers in the game which can be picked-up by a taxi. These passengers have to be delivered at theirdestination within a certain time-limit in order to score points. Taxis can steal passengers from each other as well. To make the game more interesting, there are power-ups in the city that can be picked up by the team's taxi and later on activated by the team's navigator. Sound effects are being played when certain events in the game occur. Finally a head-up display(hud) is displayed on screen to show game state info like the team's score, the game's timer and the team's identifier. When the game ends, the team with the highest score wins. Each team's final score is also submitted to the leaderboard.